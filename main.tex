\documentclass[a4paper]{article}
\usepackage[lithuanian]{babel} % LT kalbos support
\usepackage[T1]{fontenc}
\usepackage[utf8]{inputenc}
\usepackage{secdot}
\sectiondot{subsection}
\usepackage{indentfirst}
\usepackage{csquotes}
\usepackage[style=authoryear-ibid,backend=biber]{biblatex}
\bibliography{ref.bib}

\title{IBM Z vs. OpenRISC}
\author{Lukas Užomeckas 1gr. II pogr.}
\date{Gruodis, 2024}

\begin{document}

\maketitle

\section{Įvadas}
Šioje ataskaitoje apžvelgsime dvi populiarias naujojo tūkstantmečio pradžios architektūras - IBM Z ir OpenRISC. Šios architektūros buvo sukurtos siekiant skirtingų tikslų: IBM Z orientuota į patikimumą ir efektyvumą didelėse įmonių sistemose, tuo tarpu OpenRISC - į atvirumo ir lankstumo principus. Palyginę šias architektūras, galėsime įvertinti jų stipriąsias bei silpnąsias puses ir reikšmę šiandienos technologijų aplinkoje.

\printbibliography
\end{document}
